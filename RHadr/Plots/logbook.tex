============================ LOGBOOK ================================
12.Oct.02\\
I run KKMC version end of October, the same as on my web page,
and the same Stefano is using,  with cuts of email (1).
No Vacuum polarization, no FSR etc.
For 25M events I get:
     XsecY [nb] = 24.00670093 +- 0.01321713341
Comparing with 23.85 +- 0.05 nb for KKMC from Stefano,
I get more by 0.6\% +-0.2\%, which is big discrepancy (three sigma).
It may well be that we differ in the codding of the cutoffs,
but one should increase statistics before it is convincing.
(Stefano seems to have statistics or order 15M.)

When I switch on KeyELW=1 and Ihvp=3 (Burkhardt+Pietrzyk) then I get:
     XsecY [nb] = 24.57605042 +- 0.04414542441
that is more by +2.2\% +- 0.1\%, which is twice more as
as VP of Phokhara +1.1+- 0.2\% from the email (1).

REMARKS:
(A) In order to be sure that the codding of cutofs/acceptance 
is the same let me describe what I do in my code:
 Define variable TrigNew =1
 If pi+ found in /hepevt/ then
   if abs(cos(theta+) > cos(Pi*40.0/180.0) then TrigNew =0
 else TrigNew =0
 If pi- found in /hepevt/ then
   if abs(cos(theta-) > cos(Pi*40.0/180.0) then TrigNew =0
 else TrigNew =0
 If parent of pi+ or pi-       is not rho0 then  TrigNew =0
 If grandparent of pi+ or pi-  is not phi  then  TrigNew =0
 Calculate four vector VMissing = p_beam1 + p_beam2 - p_pi+ - p_pi-
 If VMissing.energy < 0.010         then TrigNew =0
 If abs(VMissing.CosTheta) < 0.010  then TrigNew =0
In this way TrigNew =1 only for events with pi+pi- comming from rho0
and passing all kinematics cuts.
Note that if I dont check for grandparent being phi then
xsection is bigger by about 1 percent
    (24.12837414 +- 0.01609173394 for phi->3pi included)

(B) Vaccum polarization I may switch on only for Ihvp=3
(Burkhardt+Pietrzyk) because the other types of VP do not work 
in the range 2m_pi<sqrt(s)<10GeV.
Wnen I do it then I find +2\% shift, as said above.
The important question: is it legal to switch VP on in KKMC?
According to discussion with Axel it is incorrect to switch on VP
near the rho, because VP is already included in the rho formfactor.
The immediate next question is: what is the meaning of the xsections
from Phokhara with and without VP?
and with which one we could compare KKMC? 
It is not clear to me, because it depends how VP is handled in the
rho formfactor in their code.

Altogether I conlude:
====================
1. We should make an effort to agree on coding cuts/acceptance in order to
eliminate the 0.6\% difference between KKMC and KKMC,
because it obscures our analysis.
2. The treatment of VP in Phokhara has to be understood before we
are sure that we are ``comparing apples with apples''.

#########################################################################
{\bf email (1)}
\begin{verbatim}
------------------------------------------------------------------------
Date: Thu, 3 Oct 2002 13:16:23 +0200 (MET DST)
From: Stefano Di Falco <Stefano.DiFalco@lnf.infn.it>
To: s.jadach@cern.ch
Subject: comparisons KKMC-Phokhara

I performed the first comparisons of the MC quantities we need in our
analysis. The results are quite good:

Cross section:
-------------
We require:
- missing energy: Emiss > 10 MeV
- missing momentum angle: 'theta_gamma' <15 or 'theta_gamma'>165 degrees
- pions angle : 40<theta_pi<140 degrees (for both pions)

Sigma(KKMC,no vacuum polarization)     = 23.85 +- 0.05 nb
Sigma(Phokhara,no vacuum polarization) = 24.06 +- 0.03 nb

                        discrepancy  0.9 +- 0.2 %

In fact we use Phokhara with vacuum polarization:
Sigma(Phokhara, vacuum polarization)   = 24.34 +- 0.02 nb
but I was not able to get a similar cross section with KKMC (switching on
the flag KeyELW in the card the cross section came out exagerately low)
------------------------------------------------------------------------
\end{verbatim}
