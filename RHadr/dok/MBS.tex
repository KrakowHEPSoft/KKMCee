%%%%%%%%%%%%%%%%%%%%%%%%%%%%%%%%%%%%%%%%%%%%%%%%%%%%%%%%%%%%%%%%%%%%%%
%%%%%%%%%%%%%%%%%%%%%%%%%%%%%%%%%%%%%%%%%%%%%%%%%%%%%%%%%%%%%%%%%%%%%%
\documentclass[dvips]{seminar}                      %%%%%%%%%%%%%%%%%%
                                                    %%%%%%%%%%%%%%%%%%
%  make tokyo-ps
%  make tokyo.ps

%%%%%%%%%%%%%%%%%%%%%%%%%%%%%%%%%%%%%%%%%%%%%%%%%%%%%%%
%%%%%%%%%%%%%%%%%%%%%%%%%%%%%%%%%%%%%%%%%%%%%%%%%%%%%%%
\begin{document}                     %%%%%%%%%%%%%%%%%%


% The upper title is defined in seminar.con One may redefine it like that
%\def\title{{\large\bf\cgr Loops and Legs, Bastei, April 2000}}

%//////////////////////////////////////////////////////////////////////////////////
%//////////////////////////////////////////////////////////////////////////////////
%//////////////////////////////////////////////////////////////////////////////////
\begin{slide}

\begin{center}
{\Large\bf\crd   Machine Beam Spread (MBS) in KKMC}\\
\vspace{2mm}
{\Large\bf\cbl   S. Jadach}\\
\vspace{2mm}
{\large\bf       Institute of Nuclear Physics, Krak\'ow, Poland}
\end{center}
%===================================================================

\vspace{1mm}
{\small\Cmar
  We discuss in this note:
  \begin{itemize}
  \item
    How MBS is implemented in KKMC?
  \item
    Is implementation of MBS in KKMC incorrect or rather inefficient?
  \item
    Danger for $w=1$ events in KKMC
  \item
    One possible fix
  \end{itemize}
}

\vfill
\end{slide}   %%%
%%%%%%%%%%%%%%%%%%



%//////////////////////////////////////////////////////////////////////////////////
%//////////////////////////////////////////////////////////////////////////////////
\begin{slide}
\yellowbox{\bf\crd MBS in KKMC was originally for LEP }

\begin{itemize}
\item
  The machine beam energy spread (MBS) was introduce in KKMC for LEP.
\item
  At the time of its implementation resonances like psi or phi we not considered.
  We had in mind Z resonance, for which the beam spread is
  much smaller than the Z resonance width.
\item
  RRes package was never tested with the nonzero beam spread,
  that is with nonzero input parameter DelEne =xpar( 2).
\end{itemize}

\vfill
\end{slide}    %%%
%%%%%%%%%%%%%%%%%%

%//////////////////////////////////////////////////////////////////////////////////
%//////////////////////////////////////////////////////////////////////////////////
\begin{slide}
\yellowbox{\bf\crd How MBS is implemented in KKMC?}

The Gaussian beam energy random variation is realized using\\
{\tt SUBROUTINE KarLud\_SmearBeams}
in the file KK2f/Karlud.f

Unfortunately the comment in front of this routine
{\footnotesize
\begin{verbatim}
*//   This is correct only for very small spread $<$ 2GeV //
\end{verbatim}}
is incorrect because:
\begin{itemize}
\item
  The MBS should be small compared to the relevant resonance width.
  The scale 2GeV is here meant to be the Z width. In reality
  for low energies MBS should be small compared the width of the low energy resonances
\item
  If the above is not true, then the actual MC implementation is not "INCORRECT"
  (although it can be in some cases, see discussion below)
  but rather it is mathematically correct, however,
  its implementation is "INEFFICIENT", from the point of view of the MC weight variation.
\end{itemize}
\vfill
\end{slide}    %%%
%%%%%%%%%%%%%%%%%%

%//////////////////////////////////////////////////////////////////////////////////
%//////////////////////////////////////////////////////////////////////////////////
\begin{slide}
\yellowbox{\bf\crd Is implementation of MBS in KKMC incorrect or inefficient?}


Let me elaborate a little bit more on the above:
In the MC algorithm for ISR (see KK2f/Karlud.f, bornv/BornV.f) we memorize
photon spectrum $f(v)=\rho_{ISR}(v)*Born(s(1-v))$.\\
In the presence of the narrow resonances
this function has peaks at $v_i=1-M_i^2/s$, where $M_i$ is mass of the resonance.\\
This $f(v)$ is memorized once for ever in the MC initialization phase.\\
In the presence of the machine beam spread (MBS) we cannot change $E=\sqrt{s}$
event-per-event, because changing s redefines $f(v)$, in particular positions $v_i$
of the peaks will move!

What I said is only half true -- we can change $s$, but
we have to compensate this change EXACTLY by the MC weight which involves
the ration $w=f(v_X)/f(v)$, where $v_X = 1-s(1-v)/s_X$ is shifted due to MBS.\\
This weight $w$ will contain in particular the ratio of two Breit-Wigners.\\
Mathematically, this solution is CORRECT, but obviously
if the beam spread is bigger than width of the resonance than the weight will go crazy.\\
One may live with the crazy MC weight and still get correct results for weighted
MC events.

\vfill
\end{slide}    %%%
%%%%%%%%%%%%%%%%%%

%//////////////////////////////////////////////////////////////////////////////////
%//////////////////////////////////////////////////////////////////////////////////
\begin{slide}
\yellowbox{\bf\crd Danger for $w=1$ events in KKMC}

What may be really dangerous is to use KKMC for MBS and narrow resonances
for  weight$=1$ events (the default)!\\
In this case events which internally have $w=$m\_WtMain $>$ m\_WTmax 
may lead to strongly distorted distributions.
(In fact the resonances may get severely distorted).

In principle one should always scrutinize and control very well statistics
of such "overweighted events", and if there are too many of them,
then one should increase parameter WTmax, see input data:
{\footnotesize
\begin{verbatim}
   9          1.0e0   WtMax =xpar( 9)   Maximum weight for rejection
 517          5.0e0   WtMax Maximum weight for rejection d-quark
\end{verbatim}}
Unfortunately, setting WtMax=1000 or so can be very wasteful, in view of the
fact that the overweighted events concentrate in a very small part of the phase space,
while we will reject 999/1000 events everywhere.\\
(Weight distribution will have  along tail -- all of the tail will 
come from the narrow resonances, which form small fraction of the total cross section!).
\vfill
\end{slide}    %%%
%%%%%%%%%%%%%%%%%%

%//////////////////////////////////////////////////////////////////////////////////
%//////////////////////////////////////////////////////////////////////////////////
\begin{slide}
\yellowbox{\bf\crd One possible fix}

However, in KKMC there is an interesting
protection mechanism for the case of many "overweighted events",
see part of code in KK2f/KK2f.f
{\footnotesize
\begin{verbatim}
         WtScaled = m_WtMain/m_WTmax
         IF( WtScaled .GT. 1d0) THEN
            m_WtMain = WtScaled
         ELSE
            m_WtMain = 1.d0
         ENDIF
\end{verbatim}}
The abnormal events with m\_WtMain $>$ m\_WTmax are still assigned the "scaled weight" 
assigned to m\_WtMain.\\
Most of events will have m\_WtMain=1, but there might be a small subset of "abnormal"
events with weight$>$1.

The possible fix is the following:
It is perhaps not very convenient, but by using
this weight WtMain event for so called ``unweighted events''
allows us safely to avoid the worst pitfalls,
leading to completely wrong result, like what you have seen.

In a sense we use then a mixture of weighted and unweighted events.
\vfill
\end{slide}    %%%
%%%%%%%%%%%%%%%%%%



%//////////////////////////////////////////////////////////////////////////////////
%//////////////////////////////////////////////////////////////////////////////////
%//////////////////////////////////////////////////////////////////////////////////
\begin{slide}

\begin{minipage}{52mm}{
\epsfig{file=../Plots/cFig0a.eps_500k_del0MeV_wt1,width=50mm}\\
\epsfig{file=../Plots/cFig0a.eps_500k_del4MeV_wt1,width=50mm}\\
\epsfig{file=../Plots/cFig0a.eps_500k_del4MeV_wted,width=50mm}
}
\end{minipage}
\begin{minipage}{52mm}
\cbl\small
\yellowbox{\bf\Color{Red}  Numerical examples}
The distribution of the effective mass $Q$ of the final state
(excluding ISR photos). All results at $\sqrt{s}=10GeV$.
All five quarks. No FSR.
Plotted regions of two lightest charm resonances.
\begin{itemize}
\item
  Upper plot for zero machine beam spread, 500k of wt=1 events.
\item
  Middle plot for machine beam spread 4MeV each beam, 500k of $wt=1$ events.
  {\crd Resonance eroded due to large amount of $wt>1$ events}.
\item
  Lower plot for machine beam spread 4MeV each beam, 500k events, MC weight $wt=1$
  for normal events and $wt>1$ weight is booked in the histogram.
  {\crd Resonance of right size but large fluctuations of the weight, see error bars!}.  
\end{itemize}
\end{minipage}
\vfill
\end{slide}   %%%
%%%%%%%%%%%%%%%%%%




%//////////////////////////////////////////////////////////////////////////////////
%//////////////////////////////////////////////////////////////////////////////////
%//////////////////////////////////////////////////////////////////////////////////
\begin{slide}
\yellowbox{\bf\crd Final remarks }

{\bf\cbl
  \begin{itemize}
  \item
    The above numerical data with psi resonance reduced in size
    seems to be consistent with what was found in BaBar test.
  \item
    Possible fix is to keep WtMain. This does not require any modification of KKMC.
  \item
    Radical solution is to introduce in KKMC the Gaussian beamstrahlung distribution
    which mimics MBS.
    This is relatively easy, but requires modification/extension of the KKMC code.
  \end{itemize}
}

\vfill
\end{slide}    %%%
%%%%%%%%%%%%%%%%%%



\end{document}                                           %%%%%%%%%%%%%%%%%%%%
%%%%%%%%%%%%%%%%%%%%%%%%%%%%%%%%%%%%%%%%%%%%%%%%%%%%%%%%%%%%%%%%%%%%%%%%%%%%%
%%%%%%%%%%%%%%%%%%%%%%%%%%%%%%%%%%%%%%%%%%%%%%%%%%%%%%%%%%%%%%%%%%%%%%%%%%%%%
