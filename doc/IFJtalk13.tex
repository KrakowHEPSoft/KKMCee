%%%%%%%%%%%%%%%%%%%%%%%%%%%%%%%%%%%%%%%%%%%%%%%%%
%%%%%  make IFJtalk13.pdf
%%%%%%%%%%%%%%%%%%%%%%%%%%%%%%%%%%%%%%%%%%%%%%%%%
\documentclass{beamer}
%\documentclass[handout]{beamer}


\mode<presentation>
{
  \usetheme{Warsaw}
 %\usetheme{Hannover}
  \setbeamercovered{transparent}
}

\usepackage[english]{babel}
\usepackage{xcolor}
\usepackage[latin1]{inputenc}

\usepackage{times}
\usepackage[T1]{fontenc}
\usepackage{listings}
%------------------------------------------------------
\usepackage{amsbsy}
\usepackage{amsmath,amssymb,bbm}
\usepackage{euscript}
\usepackage{fancybox}


%%%%%%%%%%%%%%%%%%%%%%%%%%%%%%%%%%%%%%%%%%%%%%%%%%%%%%%%%%%%%%%
%%% Macros 
\newcommand{\Pcal}{{\cal P}}
\newcommand{\Kcal}{{\cal K}}
\newcommand{\Dcal}{{\cal D}}
%
\newcommand{\Peu}{\EuScript{P}}
\newcommand{\Keu}{\EuScript{K}}
\newcommand{\Deu}{\EuScript{D}}
\newcommand{\Reu}{\EuScript{R}}
\newcommand{\Feu}{\EuScript{F}}
%
\newcommand{\Pmf}{\mathfrak{P}}
\newcommand{\Dmf}{\mathfrak{D}}
%
\newcommand{\Pbbm}{\mathbbm{P}}
\newcommand{\Rbbm}{\mathbbm{R}}
\newcommand{\Zbbm}{\mathbbm{Z}}
\newcommand{\Bbbm}{\mathbbm{B}}
\newcommand{\Pop}{\overleftarrow{\Pbbm}}
\newcommand{\Zop}{\overleftarrow{\Zbbm}}
\newcommand{\Bop}{\overleftarrow{\Bbbm}}
\newcommand{\Rop}{\overleftarrow{\Rbbm}}
%
\newcommand{\Tbf}{\mathbf{T}}
\newcommand{\Pbf}{\mathbf{P}}
\newcommand{\Dbf}{\mathbf{D}}
\newcommand{\Phibf}{\mathbf{\Phi}}
%
\newcommand{\udl}{\underline}
\newcommand{\from}{\leftarrow}
\newcommand{\bu}{\bullet}
\newcommand{\veps}{\varepsilon}
\newcommand{\Dyfs}{D_{_{\rm YFS}}}
\newcommand{\tH}{{\hat{t}}}
\newcommand{\tB}{{\bar{t}}}
\newcommand{\tBl}{{\bar{t}_\lambda}}
% boldfaces and vectors
\newcommand{\ba}{{\bf{a}}}
\newcommand{\bk}{{\bf{k}}}
\newcommand{\vkap}{{\vec{\kappa}}}
\newcommand{\vrk}{{\varkappa}}
\newcommand{\alfb}{{\bar{\alpha}}}
\newcommand{\betb}{{\bar{\beta}}}


\newcommand{\cbl}{\color{blue}}
\newcommand{\crd}{\color{red}}
\newcommand{\cmg}{\color{magenta}}
\newcommand{\cgr}{\color{green}}
\newcommand{\cwh}{\color{white}}
\newcommand{\yel}{\color{yellow}}
\newcommand{\blk}{\color{black}}
\newcommand{\cya}{\color{cyan}}

\newcommand{\ns}{\normalsize}

%%%%%%%%%%%%%%%%%%%%%%%%%%%%%%%%%%%%%%%%%%%%%%%%%%%%%%%%%%%%%%%%%%%%%%%%
%%%%%%%%%%%%%%%%%%%%%%%%%%%%%%%%%%%%%%%%%%%%%%%%%%%%%%%%%%%%%%%%%%%%%%%%
%%%%%%%%%%%%%%%%%%%%%%%%%%%%%%%%%%%%%%%%%%%%%%%%%%%%%%%%%%%%%%%%%%%%%%%%
%%%%%%%%%%%%%%%%%%%%%%%%%%%%%%%%%%%%%%%%%%%%%%%%%%%%%%%%%%%%%%%%%%%%%%%%
\title[Monte Carlo Methods] % (optional)
{ {\bf KKMC -- Status and Outlook}
} % (optional)


\author[S.~Jadach] % (optional, use only with lots of authors)
{\Large\bf S.~JADACH }


\institute[Universities of Somewhere and Elsewhere] % (optional)
{ {\large\crd IFJ-PAN, Krak\'ow, Poland}\\
  {~~~}\\
  {\footnotesize
  Partly supported by Polish Government grant\\
  {\em Narodowe Centrum Nauki} DEC-2011/03/B/ST2/02632
}}

\date[Short Occasion] % (optional)
{\small Workshop on 
   tau lepton decays: hadronic currents from data of  Belle and BaBar 
   and new physics signatures at LHC\\
   Krakow,
   September 15-20th, 2013
\vskip 4mm
 \footnotesize
  More material on
  http://jadach.web.cern.ch/
}

\subject{Talks}
% only for the PDF information catalog.

\pgfdeclareimage[height=0.5cm]{university-logo}{ifj}
\logo{\pgfuseimage{university-logo}}


\begin{document}

\begin{frame}
  \titlepage
\end{frame}
%----------------------------------------------------------------------
%----------------------------------------------------------------------
%----------------------------------------------------------------------


%----------------------------------------------------------------------
\begin{frame}[fragile]
\frametitle{\bf What is KKMC?}
%\framesubtitle{Mission statement}
\large
KKMC is the MC event generator for the process:\\
~~~~~~~~~~~~~~\fbox{$e^-e^+ \to f\bar{f}+ n\gamma$}\\
$f=\mu,\tau,u,d,s,c,b$,~~~~ $n=0,1,2...\infty$.\\
Interfaced with TAUOLA+PHOTOS\\
and with electroweak library DIZET.\\

Published version 4.13:
\begin{itemize}
\item
Computer Physics Communications 130 (2000) 360,\\
F77 code description and user guide (manual).
\item
Phys. Rev. D63 (2001) 113009,\\
physics content, CEEX exponentiations of QED corrs.\\
\end{itemize}
"Workhorse" in all 4 LEP collaborations data analysis.\\
\small
(Replacement for earlier MC's KORALZ and KORALB.)

\end{frame}
%----------------------------------------------------------------------

%----------------------------------------------------------------------
\begin{frame}[fragile]
\frametitle{Abstract (plan of the talk)}
%\framesubtitle{Mission statement}

\footnotesize
I shall talk on the present status of KKMC and
new options in version 4.22 with respect to older vers. 4.19 and 4.16 (CPC),
posted on  http://jadach.web.cern.ch/jadach/KKindex.html:

  - possibility of simulating quark beams, q+qbar -> l+lbar +photons, l=e,mu,tau, tau->X
  
  - q and qbar may have longitudinal mometum spread according to arbitrary PDFs, but zero kT.
  
(i) I was pointing out that QED effects fo q+qbar -> l+lbar process are the best, 
 and nobody can match its quality  without re-investing 10 man-years of work.
 
(ii) Spin effects, correlations  (needed for taus) are implemented in a full form 

  (including transverse correlations.)
(iii) Photos and Tauola are interfaced, albeit old versions are in the above code.

Next, I was trying to summarize for what KKMC can be useful in the LHC data analysis,
without major developments beyond the existing code:

- testing/calibrating Photos for FSR in leptonic decays of
  is an obvious thing and Zbyszek Was is doing this all the time...
  
- studies/estimations of ISR-FSR interferences in q+qbar->Z/->l+lbar data

- Electroweak+QCD corrections in the overall normalization for Z production.

- Spin correlations in Z->tau+tau, already done by Zbyszek

- what else???? Any new ideas????

\end{frame}
%----------------------------------------------------------------------


%----------------------------------------------------------------------
\begin{frame}[fragile]
\frametitle{Abstract (plan of the talk)}
%\framesubtitle{Mission statement}

\small
The discussion has concentrated mainly on the above points, however,
I have also very briefly indicated in which directions one could improve
QCD correction for the incoming beams.

In this case the upper level of KKMC would be replaced by C++ code,
which is already in place in some simple form.

The extension to q+qbar->W->l+nu is thinkable, but would require
update of the QED matrix element (EW corrs. seems to be ok)

NB. W exchange for t-channel is already there and could be exploited.
I have also mentioned work done by Bennie Ward and Scott Yost.

The bottom line in the discussion was:\\
Let us exploit the existing KKMC as much as we can for LHC
and any new idea in this direction is welcome!!!
\end{frame}
%----------------------------------------------------------------------


%----------------------------------------------------------------------
\begin{frame}[fragile]
\frametitle{Abstract (plan of the talk)}
%\framesubtitle{Mission statement}
\small
I would suggest to use version number wherever possible, a
nd more citations to related papers should be added.
First citation [16] of the KKMC version published in CPC is OK,
but quoting more precisely "KKMC version 4.13" would be better.
Moreover, the cited paper [16] in CPC contains technical description of KKMC 4.13 only,
and all physics content of KKMC v.4.13, more relevant for the discussion here,
is described in Phys. Rev. D63 (2001) 113009, so it should also be cited.

"(Although these corrections were implemented in a private version
of the program used in [20].)"
This is correct, but the description of the NNLO corrections to KKMC matrix element
relevant for THIS paper is in fact the other paper published
in parallel to ref. [20], hence:
\end{frame}
%----------------------------------------------------------------------


%----------------------------------------------------------------------
\begin{frame}[fragile]
\frametitle{Abstract (plan of the talk)}
%\framesubtitle{Mission statement}
\small
The paper S.A. Yost et.al. Acta Phys.Polon. B36 (2005) 2379-2386 should be cited along with [20]
Moreover, NNLO corrections in [20] and inthis paper are shown in numerical form only,
The complete "algebraic" description of the NNLO formulas has been
published later on, in Phys.Rev. D73 (2006) 073001, so I recommend to cite it also.
(an extension of the previous work in Phys.Rev. D65 (2002) 073030 --
citation of this paper is my suggestion only, not recommendation.).
Unfortunately KKMC with complete NNLO corrections is not yet public.
Nevertheless the above related works would be helpful for the reader.
\end{frame}
%----------------------------------------------------------------------


%----------------------------------------------------------------------
\begin{frame}[fragile]
\frametitle{Abstract (plan of the talk)}
%\framesubtitle{Mission statement}
\small
"Furthermore, the present version of the program is restricted to leptonic final states
and thus cannot serve for the analysis of the multitude of hadronic states mentioned above"
is not very precise. It is better to say:
"the above version 4.13 of the KKMC program...".
However, even this statement is still not precise,
because the restriction to leptonic states is only true for KKMC version 4.13
More precisely, in the unpublished (but public) version 4.16 of KKMC, posted since 2002 on
http://jadach.web.cern.ch/jadach/KKindex.html
implements a multitude of the final multipion resonances and nonresonant background,
albeit with not so very high quality of the decay simulation,
and with the QED ISR matrix element limited to NLO.
My suggestion is to mention this version 4.16 of KKMC at this point,
for instance in the footnote or in the literature.
\end{frame}
%----------------------------------------------------------------------


%----------------------------------------------------------------------
\begin{frame}[fragile]
\frametitle{Abstract (plan of the talk)}
%\framesubtitle{Mission statement}
\footnotesize
KKMC Monte Carlo for fermion pair production at electron-positron collision. 
Source code is below:

Production Version 4.16, October 30-th, 2001,  
KKMC-v.4.16d-export.tar.gz Improved nu-nu. RRes for gamma* --> narrow resonances, NEW!

Developement Version 4.19, September  27-th, 2002,  
KKMC-v.4.19.b-export.tar.gz Improved  RRes for low energy colliders, 
C++ wrapper and corrected nu-nu matrix element. 
ISR with complete NLL corrections, Phys.Rev. D65(2002)073030 by S.J., M.Mells, B.F.L.Ward and S.A. Yost

Developement Version 4.22, June  15-th, 2013,  
KKMC\_v4\_22.tgz Tested mu+mu- and q-qbar beams (instead of e+e-) at fixed energy. 
Moreover, q-qbar collinear PDFs instead of beamstrahlung 
as a patch in the source code (temporary solution)

The precision Monte Carlo event generator KK for two-fermion final  states in e+ e- collisions,  
Comput. Phys. Commun. 130 (2000) 260, hep-ph/9912214.

Coherent exclusive exponentiation for precision Monte Carlo  calculations,  
Phys. Rev. D63 (2001) 113009, hep-ph/0006359

\end{frame}
%----------------------------------------------------------------------


%----------------------------------------------------------------------
\begin{frame}[fragile]
\frametitle{Abstract (plan of the talk)}
%\framesubtitle{Mission statement}
Summary of KK features in comparison with KORALZ   and KORALB
Physical and technical precision for Born-like process and Z radiative return  EEX ,  200GeV
Charge asymmetry , angular distributions, and  total cross section     at 189GeV,  CEEX versus KORALZ
Charge asymmetry , angular distributions, and  total cross section     at 120GeV,  CEEX versus KORALZ
Charge asymmetry , angular distributions, and  total cross section     at  Z mass,  CEEX versus KORALZ
Charge asymmetry and  total cross section     at 500GeV,  CEEX only
ISR*FSR  contribution to total cross section   close to Z mass (LEP1) for leptons and quarks,  CEEX

\end{frame}
%----------------------------------------------------------------------


\end{document}
%%%%%%%%%%%%%%%%%%%%%%%%%%%%%%%%%%%%%%%%%%%%%%%%%%%%%%%%%%%%%%%%%%%%%%%%
%%%%%%%%%%%%%%%%%%%%%%%%%%%%%%%%%%%%%%%%%%%%%%%%%%%%%%%%%%%%%%%%%%%%%%%%
%%%%%%%%%%%%%%%%%%%%%%%%%%%%%%%%%%%%%%%%%%%%%%%%%%%%%%%%%%%%%%%%%%%%%%%%
%%%%%%%%%%%%%%%%%%%%%%%%%%%%%%%%%%%%%%%%%%%%%%%%%%%%%%%%%%%%%%%%%%%%%%%%


%----------------------------------------------------------------------
\begin{frame}[fragile]
\frametitle{Abstract (plan of the talk)}
%\framesubtitle{Mission statement}
...
\end{frame}
%----------------------------------------------------------------------


